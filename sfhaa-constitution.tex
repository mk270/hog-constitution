\documentclass[10pt]{mk-articles-of-association}

\newcommand{\EC}[0]{Board}
\newcommand{\Exec}[0]{\EC{} }
\newcommand{\RA}[0]{Residents' Organisation}
\newcommand{\RTA}[0]{Recognised Tenants' Association}
\newcommand{\LAFA}[0]{Leasehold and Freehold Abuses}

\newcommand{\Name}[0]{Housing Accountability Limited}

\begin{document}

\title{
Articles of Association\\
}
\author{\Name{}\\
}
\date{}

\maketitle

\begin{center}
Company No 13176148\\
(private company limited by guarantee)\\
\medskip
Incorporated: 1 February 2021 \\
\medskip
Articles updated by special resolution on 21~February~2021 and 29~March~2022
\end{center}



\section{Applicability of Model Articles}

\begin{constenum}

\item The Model Articles provided for in law shall apply to the
  Association, except insofar as they are modified, excluded or
  contradicted by this Constitution.

\item The version of the Model Articles which shall apply is those
  which were in force for a private company limited by guarantee on
  30 January 2021; all other references in this Constitution
  to any act or legislation shall include any statutory modification
  or re-enactment of it for the time being in force.

\item The following Model Articles shall not apply to the Association:
  \begin{constenum}
  \item Model Article 2 --- liability of members
  \item Model Article 16 --- rules
  \item Model Article 21 --- applications for membership
  \item Model Article 35 --- company seals
  \item Model Article 36 --- inspections
  \end{constenum}

\item The following Model Article provisions shall apply to the Association
  as though for ``member'' they stated ``Full Member'':

  \begin{constenum}
  \item Model Article 2 --- liability of members
  \item Model Article 4 --- reserve powers
  \item Model Article 25 --- chairing general meetings
  \end{constenum}

\end{constenum}

\section{Definitions}

Unless the context indicates another meaning, words or expressions
defined in these Articles shall bear the same meaning as in the
Companies Act 2006 as it was in force on the date when the company was
incorporated:

  \begin{constenum}

    \definition{Association}{the company established with these articles};

    \definition{\EC{}}{the directors of the Association};

    \definition{Catchment Area}{England and Wales};

    \definition{clear days}{a set of consecutive days between two events, not
      including the days on which the events themselves shall occur};

    \definition{Constitution}{these articles of association};

    \definition{fleecehold}{the state of affairs wherein the owner of
      a freehold house is required, by a covenant, rentcharge or
      similar mechanism restricting the freehold title, to follow
      regulations made by, and pay service charges to, a private
      entity that is not controlled exclusively by those subject to
      the said regulations and service charges};

    \definition{Housing Problem}{any problem affecting housing,
      particularly build defects, cladding issues,
      \LAFA, and service charge disputes};

    \definition{leaseholder}{a natural person who, either solely, or
      jointly with other persons, holds a long lease on residential
      property within the Catchment Area};

    \definition{long lease}{a lease initially granted for a term of
      twenty-one years or greater};

    \definition{natural person}{a living human being over the age of eighteen
      years};

    \definition{private residential freeholder}{a natural person who,
        either solely, or jointly with other persons, holds the freehold
        of a house subject to fleecehold within the Catchment Area};

    \definition{\RTA{}}{an organisation which has been recognised for
      the purposes of section 29 of the Landlord and Tenant Act 1985 by
      the relevant freeholder or tribunal};

    \definition{\RA}{a membership organisation
      \begin{constenum}
        \item generally comprising residents or others with a connection
          to a definite location larger than a block of flats,
        \item whose purposes effectively include the advancement of
          the interests of its members by virtue of their common
          connection to the location;
      \end{constenum}
    }

    \definition{Right to Manage company}{a company which
      \begin{constenum}
        \item is an ``RTM company'' as defined in section 73 of the
          Commonhold and Leasehold Reform Act 2002, \ITand
        \item which is controlled by the leaseholders of said
          premises, or at least such subset as are its members;
      \end{constenum}
    }

    \definition{RTA}{\RTA};

    \definition{unitholders}{leaseholders or private residential
      freeholders}; \ITand

    \definition{UK}{the United Kingdom of Great Britain and Northern Ireland}.

  \end{constenum}



\section{Objects}
\begin{constenum}

\item The Objects of the Association shall be:

  \begin{constenum}

  \item the phasing out of residential leasehold within the Catchment
    Area, with compensation only for the freeholders' legitimate
    property interests,

  \item the comprehensive reform of residential leasehold while it
    continues to exist,

  \item the promotion of commonhold as the replacement for residential
    leasehold,

  \item the elimination of fleecehold and other \LAFA{} within the
    Catchment Area,

  \item the resolution or mitigation of particular instances of such
    Abuses and other Housing Problems in the Catchment Area,

  \item the fostering of collaboration between those dealing with
    Housing Problems across the Catchment Area,

  \item the education and training of its Members and others on
    Housing Problems, and the conduct of related research, \ITand

  \item the promotion of the interests, both of the Members and the
    general public in the UK, in relation to Housing Problems.

  \end{constenum}

\newpage

\item In the furtherance of its Objects, the Association shall
\begin{constenum}

  \item collect, organise, store and publish information about
    Housing Problems in the Catchment Area;

  \item solicit information towards and maintain Registers of
    \begin{constenum}
    \item \LAFA, \ITand
    \item instances of such Abuses and other Housing Problems within
      Catchment Area,
    \item proposed reforms to alleviate said Problems;
    \end{constenum}

  \item encourage the formation of, registration of and participation
    in \RTA{}s where such bodies do not already exist, or where they
    are not registered with the relevant freeholder, and to do the same
    in relation to RTM companies;

  \item liaise with, and refer Housing Problems to: police, regulators,
    press, media, politicians and other appropriate individuals and bodies;

  \item develop IT systems;

  \item network, collaborate and cooperate with those outside the
    Catchment Area facing similar concerns;

  \item do all other such lawful things as conduce to the attainment
    of the Association's Objects.
\end{constenum}

\item For the purposes of this article, the legitimate property interests of
  the freeholder in a given estate shall be held to exclude:
  \begin{constenum}
    \item the monetary value of future variable service charge income
      (and of future administration charge income) in excess of what
      that estate's unitholders would pay if they could freely choose
      the manager of the services; \ITand
    \item in the context of forfeiture, the value of the remaining equity in a
      leaseholder's property after settlement of debts to the freeholder or
      other parties to the lease.
  \end{constenum}

\item Except in so far as it would indirectly further the Association's own
  Objects, the Association shall refrain from involvement in or the conduct of
  \begin{constenum}
  \item planning applications as such, except in so far as they relate
    to Housing Problems;
  \item the core activities of \RA{}s, except in so far as they
    relate to Housing Problems;
  \end{constenum}

\item Even where it would further the Association's Objects, the Association
  shall refrain from involvement in or the conduct of
  \begin{constenum}
  \item social events; \ITand
  \item community building.
  \end{constenum}

\item \avoiddoubt, discrimination on any basis generally held to be
  impermissible in the UK is contrary to the Objects of the
  Association.

\item The Association shall not be suffered to operate in
  a party-political manner.

\end{constenum}



\section{Non-Distribution}

  The income and property of the Association howsoever derived shall be
  applied solely in promoting the Association’s Objects. No distribution shall
  be paid or capital otherwise returned to the Members in cash or otherwise.
  However, nothing in this Constitution shall prevent any payment in good
  faith by the Association of:
  \begin{constenum}
  \item duly authorised, reasonable and proper remuneration to any Member,
    officer or servant of the Association for any services rendered to the
    Association;
  \item any interest on money lent by any Member at a reasonable and proper
    rate; \ITor
  \item reasonable out-of-pocket expenses properly incurred by a Member.
  \end{constenum}



\section{\Exec}
  \begin{constenum}

  \item The Association shall have a governing body which shall be
    known as the ``\EC{}'', or ``Board of Directors''. Individual
    members of the \Exec may be known as Directors and shall be
    registered with Companies House as the directors of the company
    established by these Articles, which constitutes the Association
    for the purposes of UK law.

  \item The \Exec shall decide the overall strategy and priorities of
    the Association, and shall have any other responsibilities
    necessitated or implied by this Constitution and not otherwise
    allocated to someone or something else.

  \item The \Exec shall have, and
    shall not delegate or assign, the exclusive power to do the
    following:\label{nondelegation}

    \begin{constenum}
      \item making Bye Laws for the Association;

      \item the expulsion or suspension of Members;

      \item the creation, abolition and modification of the membership of
        any committees or sub-committees as it sees fit;

      \item the appointment of trustees for bank accounts, assets and
        digital and online systems; \ITand

      \item co-opting additional Full Members as members of the \EC{}.

    \end{constenum}

  \item The \Exec may alter the name of the Association.

  \item Decision-making by the \Exec shall be by majority vote, and each
    member shall have only one vote.\label{boardvote}

  \item The Quorum for meetings of the \Exec shall not be less than
    half the size of its membership. Additionally, the Quorum shall
    not be fewer than three, unless there are fewer than three
    members. Where there are fewer than three members, the Quorum shall
    be the same as the size of the membership.

  \item Only natural persons who are Members of the Association
    may be members of the \EC{}.

  \item Any member of the \Exec who ceases to be Member of the Association
    shall automatically cease to be a member of the \EC{}.

  \item The \Exec shall not suffer the time or other resources of
    the Association to be applied otherwise than in the furtherance of
    the Association's Objects, nor shall they suffer unauthorised individuals
    to purport to represent the Association.

\end{constenum}


\section{Membership}
  \begin{constenum}

  \item There shall be three classes of membership, which shall be mutually
    exclusive:
    \begin{constenum}
      \item Full Members,
      \item Honorary Members, \ITand
      \item Registered Supporters.
    \end{constenum}

  \item Membership shall not be transferable to anyone or anything else.

  \item Members who were Voting Members before the Constitution was
    amended with the effect of renaming that class of membership shall
    be treated as having become Full Members by virtue of the said
    amendment.

  \item Subject to any exceptions provided for by virtue of
    \articleref{councils} or \articleref{boardvote},
    Registered Supporters and
    Honorary Members shall not be entitled to vote.

  \item The \Exec shall determine the class of membership of a Member
    or potential Member at the point at which that Member is admitted
    to the Association.  The class of membership enjoyed by a Member
    shall not be changed without the written agreement both of that
    Member and of the \EC{}, except that the \Exec may, within sixty
    clear days of becoming aware that a Full Member is no longer a
    unitholder, re-register a Full Member as a member of a different
    class.

\end{constenum}



\section{Qualifications for Membership}

\begin{constenum}

  \item Membership shall be open to other individuals or organisations that
    \begin{constenum}
      \item make an application to join in the form required by the \EC, \ITand
      \item are approved by the \EC{}, at its sole discretion.
    \end{constenum}

  \item Only natural persons shall be admitted to Honorary Membership.

  \item Membership of the Association shall be restricted to
    \begin{constenum}
    \item natural persons,
    \item \RTA{}s, \ITand
    \item Right To Manage companies.
    \end{constenum}

  \item It shall be a qualification for Membership that a Member
    \begin{constenum}

    \item agree to be a Member and abide by the requirements of
      Membership, including agreement to provide contact details and
      be bound by this Constitution,

    \item support the Objects of the Association, free of any conflict
      of interest, \ITand

    \item have the legal form of a natural person, an unincorporated
      association or a UK-registered company limited by guarantee.

    \end{constenum}

  \item For the purposes of determining if an organisation is
    qualified to be a Member, the \Exec may treat as though it were a
    \RTA{} any organisation which in the opinion of the \Exec is a
    \textit{bona fide} potential RTA but which is not so recognised by
    the relevant freeholder.

  \item An applicant for membership who is a natural person may not be
    admitted as a Full Member unless the applicant is a unitholder.

\end{constenum}



\section{Termination or Refusal of Membership}

\begin{constenum}

  \item
    Membership shall cease when
    \begin{constenum}
      \item a Member dies or ceases to exist,
      \item when he, she or it no longer qualifies to be a Member,
      \item when expelled pursuant to \articleref{expulsion},
      \item when his, her or its Membership lapses pursuant to
        \articleref{lapse},
      \item when the Member resigns by written notice to the
        Association unless, after the resignation, there would be
        fewer than two Full Members, \ITor{}
      \item any sum due from the Member to the Association is not paid in full
        within six months of it falling due.
    \end{constenum}

\item Membership shall be terminated if
\label{expulsion}
  the Member is removed from Membership by a resolution of the
  \Exec that it is in the best interests of the Association that his or
  her Membership be terminated. A resolution to remove a Member from
  Membership may only be passed if

\begin{constenum}

\item the Member has been given at least twenty-one clear days' notice in
  writing of the meeting of the \Exec at which the resolution will
  be proposed and the reasons why it is to be proposed,

\item the Member or, at the option of the Member, the Member's
  representative (who need not be a Member of the Association) has been
  allowed to make representations to the meeting.

\end{constenum}

\item The \Exec may deem a Member's membership to have lapsed where
  the Member has, despite reasonable efforts by the Association, not
  responded to contact attempts for more than six months.
  \label{lapse}

\end{constenum}




\section{Councils}
\label{councils}

The \EC{} may establish by Bye Laws bodies which
  shall be styled as ``Councils'' to represent particular interests, and
  such Bye Laws may provide for such Councils to appoint persons to
  the \EC{}.


\newpage

\section{Bye Laws}

\begin{constenum}

\item The \Exec may from time to time make such reasonable and proper
  rules, to be termed ``Bye Laws'', as they may deem necessary or
  expedient for the proper conduct and management of the Association.

\item Any additional rules created by the \Exec as Bye Laws
  shall be maintained as a single document, and shall be
  consistent with the letter and spirit of this Constitution, and,
  particularly when concerned with disputes between Members,
  consistent with generally accepted principles of Due Process and
  Natural Justice.

\item The Bye Laws may regulate the following matters but shall not be
  restricted to them:

\begin{constenum}

\item the admission of Members of the Association (including the admission
  of organisations to Membership) and the rights and privileges of
  such Members, and the entrance fees, subscriptions and other fees or
  payments to be made by Members;

\item the arrangements for the election or appointment of members of
  the \Exec by any Councils pursuant to
  \articleref{councils};

\item the conduct of Members of the Association in relation to one
  another, and to the Association's employees and volunteers;

\item the procedure at general meetings and meetings of the \Exec and
  the Councils in so far as such procedure is not regulated by the
  Companies Act or by these Articles;

\item the format and contents of the Register of \LAFA, such that
  it shall provide a general means of enumerating, identifying and
  discussing \LAFA{} both within and outside the Association; \ITand

\item generally, all such matters as are commonly the subject matter
  of company rules.

\end{constenum}

\item The Association in general meeting shall have the power to
  alter, add to or repeal the rules or Bye Laws.

\item The \Exec shall adopt such means as they think sufficient to
  bring the rules and Bye Laws to the notice of Members of the Association.

\item The Bye Laws shall be binding on all Members of the
  Association. No Bye Law shall be inconsistent with, or shall
  alter, repeal or subvert anything contained in these Articles or
  the Companies Act 2006.
\end{constenum}


\newpage

\section{Winding Up}

On the merger of the Association with any other body, or the
  winding-up or dissolution of the Association, after provision has
  been made for all its debts and liabilities, any assets or property
  that remains available to be distributed or paid shall not be paid
  or distributed to the Members (except to a Member that qualifies
  under this Article) but shall be transferred to Leasehold Knowledge
  Partnership Ltd (company number: 08999652; charity number: 1162584).


\end{document}
